
\documentclass[a4paper,10pt]{article}
\usepackage{ucs}
\usepackage[utf8]{inputenc}
\usepackage[brazil]{babel}
\usepackage{a4wide}
\usepackage{fontenc}
\usepackage{times}
\usepackage{indentfirst}
\usepackage{graphicx,tabularx}
\usepackage[]{hyperref}
\sloppy
\date{Data de processamento: \today}\begin{document}
\author{Diretoria de Avaliação Institucional (DIAVI) \\ Universidade Federal de Juiz de Fora}

\title{RELATÓRIO DE RESULTADOS DA AVALIAÇÃO DE DEPARTAMENTO\\ Código do Departamento: CCOGV}
\maketitle
\section{INTRODUÇÃO}
Este relatório objetiva apresentar os resultados da avaliação de disciplinas do Departamento     de código CCOGV da Universidade Federal de Juiz de Fora, realizada pela     Diretoria de Avaliação Institucional e os encaminhamentos propostos a     partir destes resultados.

\begin{center}
\begin{tabularx}{\linewidth}{X|l}

Público-alvo:& Departamento  CCOGV\\

Período de coleta de dados:& 2019/3.\\

Forma de aplicação:& Online, por meio do SIGA.\\

Docentes respondentes:& 7\\

Número de disciplinas do departamento avaliadas pelos Docentes:& 13\\

Alunos   respondentes:& 77\\

Número de disciplinas do departamento  avaliadas pelos   Alunos:& 13\\
\end{tabularx}
\end{center}

Os resultados relativos as respostas dos discentes serão omitidos na ausência de alunos respondentes, ou na eventualidade de somente um aluno responder o questionário discente.
\section{MÉTODOS}
Este relatório se refere ao período 2019/3, com base em dados     coletados através da aplicação de instrumentos de avaliação via SIGA     implementados pela Diretoria de Avaliação Institucional (DIAVI) da UFJF, em atendimento     ao que estabelece a Lei Sinais e a Resolução Consu 13/2015 (UFJF),     com objetivo de contribuir para a avaliação própria do departamento CCOGV.    Foram aplicados um instrumento para discentes e outro para docentes, ambos contendo     15 questões versando sobre as disciplinas na modalidade presencial oferecidas pela UFJF no     referido período, visando, especificamente, coletar impressões sobre: atuação docente, atuação discente,     recursos empregados, qualidade da disciplina ministrada.     As respostas foram colhidas      com participação espontânea e garantia de    sigilo de participantes e avaliados.
\section{QUESTÕES APRESENTADAS NOS FORMULÁRIOS}
As seguintes questões foram {\bf objeto de avaliação pelos discentes} através do SIGA.

\begin{center}
\small{
\begin{tabularx}{\linewidth}{l|X}
Q01&Sua motivação para estudar esta disciplina foi alta.\\\\
Q02&Você teve facilidade para acompanhar as atividades da disciplina.\\\\
Q03&Seu empenho durante a disciplina foi elevado.\\\\
Q04&O plano de ensino apresentado pelo(a) docente contém: ementa, objetivos, metodologias de ensino, critérios de avaliação, conteúdos e bibliografia da disciplina.\\\\
Q05&Os conteúdos trabalhados na disciplina foram coerentes com os que foram apresentados no plano de ensino.\\\\
Q06&O(A) docente deu retorno sobre todas as avaliações realizadas: correções, discussão dos pontos principais, esclarecimentos sobre os erros cometidos.\\\\
Q07&Os recursos didáticos, audiovisuais, tecnológicos empregados pelo(a) docente contribuíram para que a aula fosse mais produtiva.\\\\
Q08&O(A) docente cumpriu a carga horária prevista no semestre para a disciplina.\\\\
Q09&O grau de dificuldade das avaliações foi compatível com as aulas ministradas.\\\\
Q10&A disciplina contribuiu na preparação para o exercício profissional.\\\\
Q11&A disciplina contribuiu para a aquisição de formação teórica/prática na área. \\\\
Q12&A disciplina contribuiu para a aquisição de cultura geral.\\\\
Q13&O(A) docente dispôs de horários de atendimento extra-classe.\\\\
Q14& As notas de todas as avaliações parciais foram divulgadas no SIGA até 3 (três) dias antes da data da avaliação subsequente, conforme determina o RAG. \\\\
Q15&O(A) docente manteve um comportamento respeitoso e cortês em relação aos discentes.
\end{tabularx}
}
\end{center}
As questões abaixo foram {\bf objeto de avaliação pelos docentes} através do SIGA.

\begin{center}
\small{
\begin{tabularx}{\linewidth}{c|X}
Q01&A turma se mostrou motivada para estudar a disciplina.\\\\
Q02&A turma teve facilidade para acompanhar acompanhar as atividades da disciplina\\\\
Q03&A turma se empenhou em acompanhar as atividades da disciplina.\\\\
Q04&Você apresentou plano de ensino contendo: ementa, objetivos, metodologias de ensino, critérios de avaliação, conteúdos e bibliografia da disciplina.\\\\
Q05&Os conteúdos trabalhados na disciplina foram coerentes com os que foram apresentados no plano de ensino.\\\\
Q06&Você deu retorno sobre todas as avaliações realizadas: correções, discussão dos pontos principais, esclarecimentos sobre os erros cometidos.\\\\
Q07&Os recursos didáticos, audiovisuais, tecnológicos empregados pelo(a) docente contribuíram para que a aula fosse mais produtiva.\\\\
Q08&Você cumpriu a carga horária prevista no semestre para a disciplina.\\\\
Q09&O grau de dificuldade das avaliações foi compatível com as aulas ministradas.\\\\
Q10&A disciplina contribuiu na preparação para o exercício profissional.\\\\
Q11&A disciplina contribuiu para a aquisição de formação teórica/prática na área. \\\\
Q12&A disciplina contribuiu para a aquisição de cultura geral.\\\\
Q13&Você dispôs de horário de atendimento extra-classe.\\\\
Q14& As notas de todas as avaliações parciais foram divulgadas no SIGA até 3 (três) dias antes da data da avaliação subsequente, conforme determina o RAG. \\\\
Q15&Você manteve um comportamento respeitoso e cortês em relação aos discentes.
\end{tabularx}
}
\end{center}
\section{RESPOSTAS}
As questões podem ser respondidas com um número de 1 a 5 numa escala desde {\it Discordo Totalmente} (1) a {\it Concordo Totalmente} (5). Não sendo permitidas múltiplas respostas e sendo possível a alteração antes do envio do formulário. O valor 0 (zero) indica {\it Não se Aplica}.

Os nomes das disciplinas foram alterados para garantir a segurança da informações de modo que não seja possível a identificação das turmas, dos alunos  e dos professores que responderam a avaliação.

\begin{figure}[h]
\centering
\includegraphics[width=0.85\linewidth]{analise_geral_departamento_CCOGV_docentes.png}
\includegraphics[width=0.85\linewidth]{analise_geral_departamento_CCOGV_alunos.png}
\caption{\label{fig:analise_geral_departamento}            Panorama geral das respostas de todas as  disciplinas do departamenbto para as questões apresentadas.}
\end{figure}

\begin{figure}[h]
\centering
\includegraphics[width=0.999\linewidth]{matriz_corr__CCOGV_docentes.png}
\caption{\label{fig:corr_docentes}Correlação das respostas dos professores do departamento CCOGV. Entradas com em branco indicam que não havia informação suficiente para o cálculo das correlações.}
\end{figure}

\begin{figure}[h]
\centering
\includegraphics[width=0.999\linewidth]{matriz_corr__CCOGV_alunos.png}
\caption{\label{fig:corr_alunos}Correlação das respostas dos alunos para o departamento CCOGV. Entradas com em branco indicam que não havia informação suficiente para o cálculo das correlações.}
\end{figure}
\begin{figure}[h]
\centering
\includegraphics[width=0.485\linewidth]{analise_disciplina_departamento_CCOGV_4FF7C15B3C6B6DB4F118BCC6D4BFD6FC_docentes.png}
\includegraphics[width=0.485\linewidth]{analise_disciplina_departamento_CCOGV_4FF7C15B3C6B6DB4F118BCC6D4BFD6FC_alunos.png}
\caption{\label{fig:analise_geral_departamento}                Distribuição das respostas para a disciplina 4FF7C15B3C6B6DB4F118BCC6D4BFD6FC. }
\end{figure}
\begin{figure}[h]
\centering
\includegraphics[width=0.485\linewidth]{analise_disciplina_departamento_CCOGV_BD817A1E5C92DCE2DEE510422C1E1E7E_docentes.png}
\includegraphics[width=0.485\linewidth]{analise_disciplina_departamento_CCOGV_BD817A1E5C92DCE2DEE510422C1E1E7E_alunos.png}
\caption{\label{fig:analise_geral_departamento}                Distribuição das respostas para a disciplina BD817A1E5C92DCE2DEE510422C1E1E7E. }
\end{figure}
\begin{figure}[h]
\centering
\includegraphics[width=0.485\linewidth]{analise_disciplina_departamento_CCOGV_23EB2057363EBF496FB97534A2CBA6DC_docentes.png}
\includegraphics[width=0.485\linewidth]{analise_disciplina_departamento_CCOGV_23EB2057363EBF496FB97534A2CBA6DC_alunos.png}
\caption{\label{fig:analise_geral_departamento}                Distribuição das respostas para a disciplina 23EB2057363EBF496FB97534A2CBA6DC. }
\end{figure}
\begin{figure}[h]
\centering
\includegraphics[width=0.485\linewidth]{analise_disciplina_departamento_CCOGV_AFB74FD50B8898CC3189595800C28B2B_docentes.png}
\includegraphics[width=0.485\linewidth]{analise_disciplina_departamento_CCOGV_AFB74FD50B8898CC3189595800C28B2B_alunos.png}
\caption{\label{fig:analise_geral_departamento}                Distribuição das respostas para a disciplina AFB74FD50B8898CC3189595800C28B2B. }
\end{figure}
\begin{figure}[h]
\centering
\includegraphics[width=0.485\linewidth]{analise_disciplina_departamento_CCOGV_D260E7FAE02929DE5AAA296B6971B7C1_docentes.png}
\includegraphics[width=0.485\linewidth]{analise_disciplina_departamento_CCOGV_D260E7FAE02929DE5AAA296B6971B7C1_alunos.png}
\caption{\label{fig:analise_geral_departamento}                Distribuição das respostas para a disciplina D260E7FAE02929DE5AAA296B6971B7C1. }
\end{figure}
\begin{figure}[h]
\centering
\includegraphics[width=0.485\linewidth]{analise_disciplina_departamento_CCOGV_D2AFBB19111CC877B855E2B7BD347359_docentes.png}
\includegraphics[width=0.485\linewidth]{analise_disciplina_departamento_CCOGV_D2AFBB19111CC877B855E2B7BD347359_alunos.png}
\caption{\label{fig:analise_geral_departamento}                Distribuição das respostas para a disciplina D2AFBB19111CC877B855E2B7BD347359. }
\end{figure}
\begin{figure}[h]
\centering
\includegraphics[width=0.485\linewidth]{analise_disciplina_departamento_CCOGV_9465B3845A00AEBE11272EED77B860FB_docentes.png}
\includegraphics[width=0.485\linewidth]{analise_disciplina_departamento_CCOGV_9465B3845A00AEBE11272EED77B860FB_alunos.png}
\caption{\label{fig:analise_geral_departamento}                Distribuição das respostas para a disciplina 9465B3845A00AEBE11272EED77B860FB. }
\end{figure}
\begin{figure}[h]
\centering
\includegraphics[width=0.485\linewidth]{analise_disciplina_departamento_CCOGV_51042ABFE3038087429DF01BE51F656F_docentes.png}
\includegraphics[width=0.485\linewidth]{analise_disciplina_departamento_CCOGV_51042ABFE3038087429DF01BE51F656F_alunos.png}
\caption{\label{fig:analise_geral_departamento}                Distribuição das respostas para a disciplina 51042ABFE3038087429DF01BE51F656F. }
\end{figure}
\begin{figure}[h]
\centering
\includegraphics[width=0.485\linewidth]{analise_disciplina_departamento_CCOGV_043EF65836FD4E5D090B30663E731F5F_docentes.png}
\includegraphics[width=0.485\linewidth]{analise_disciplina_departamento_CCOGV_043EF65836FD4E5D090B30663E731F5F_alunos.png}
\caption{\label{fig:analise_geral_departamento}                Distribuição das respostas para a disciplina 043EF65836FD4E5D090B30663E731F5F. }
\end{figure}
\begin{figure}[h]
\centering
\includegraphics[width=0.485\linewidth]{analise_disciplina_departamento_CCOGV_50834D252F0067F4CE4CDDDD2C189E4C_docentes.png}
\includegraphics[width=0.485\linewidth]{analise_disciplina_departamento_CCOGV_50834D252F0067F4CE4CDDDD2C189E4C_alunos.png}
\caption{\label{fig:analise_geral_departamento}                Distribuição das respostas para a disciplina 50834D252F0067F4CE4CDDDD2C189E4C. }
\end{figure}
\begin{figure}[h]
\centering
\includegraphics[width=0.485\linewidth]{analise_disciplina_departamento_CCOGV_C1507584A37DCE4651D9D2704EA0A974_docentes.png}
\includegraphics[width=0.485\linewidth]{analise_disciplina_departamento_CCOGV_C1507584A37DCE4651D9D2704EA0A974_alunos.png}
\caption{\label{fig:analise_geral_departamento}                Distribuição das respostas para a disciplina C1507584A37DCE4651D9D2704EA0A974. }
\end{figure}
\begin{figure}[h]
\centering
\includegraphics[width=0.485\linewidth]{analise_disciplina_departamento_CCOGV_173DB5294A538B116DDEEC44C4BC8408_docentes.png}
\includegraphics[width=0.485\linewidth]{analise_disciplina_departamento_CCOGV_173DB5294A538B116DDEEC44C4BC8408_alunos.png}
\caption{\label{fig:analise_geral_departamento}                Distribuição das respostas para a disciplina 173DB5294A538B116DDEEC44C4BC8408. }
\end{figure}
\begin{figure}[h]
\centering
\includegraphics[width=0.485\linewidth]{analise_disciplina_departamento_CCOGV_88D704AB2FFD54B84D3B43D08BEE5345_docentes.png}
\includegraphics[width=0.485\linewidth]{analise_disciplina_departamento_CCOGV_88D704AB2FFD54B84D3B43D08BEE5345_alunos.png}
\caption{\label{fig:analise_geral_departamento}                Distribuição das respostas para a disciplina 88D704AB2FFD54B84D3B43D08BEE5345. }
\end{figure}

\end{document}
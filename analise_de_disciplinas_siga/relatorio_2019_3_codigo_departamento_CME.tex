
\documentclass[a4paper,10pt]{article}
\usepackage{ucs}
\usepackage[utf8]{inputenc}
\usepackage[brazil]{babel}
\usepackage{a4wide}
\usepackage{fontenc}
\usepackage{times}
\usepackage{indentfirst}
\usepackage{graphicx,tabularx}
\usepackage[]{hyperref}
\sloppy
\date{Data de processamento: \today}\begin{document}
\author{Diretoria de Avaliação Institucional (DIAVI) \\ Universidade Federal de Juiz de Fora}

\title{RELATÓRIO DE RESULTADOS DA AVALIAÇÃO DE DEPARTAMENTO\\ Código do Departamento: CME}
\maketitle
\section{INTRODUÇÃO}
Este relatório objetiva apresentar os resultados da avaliação de disciplinas do Departamento     de código CME da Universidade Federal de Juiz de Fora, realizada pela     Diretoria de Avaliação Institucional e os encaminhamentos propostos a     partir destes resultados.

\begin{center}
\begin{tabularx}{\linewidth}{X|l}

Público-alvo:& Departamento  CME\\

Período de coleta de dados:& 2019/3.\\

Forma de aplicação:& Online, por meio do SIGA.\\

Docentes respondentes:& 23\\

Número de disciplinas do departamento avaliadas pelos Docentes:& 16\\

Alunos   respondentes:& 100\\

Número de disciplinas do departamento  avaliadas pelos   Alunos:& 14\\
\end{tabularx}
\end{center}

Os resultados relativos as respostas dos discentes serão omitidos na ausência de alunos respondentes, ou na eventualidade de somente um aluno responder o questionário discente.
\section{MÉTODOS}
Este relatório se refere ao período 2019/3, com base em dados     coletados através da aplicação de instrumentos de avaliação via SIGA     implementados pela Diretoria de Avaliação Institucional (DIAVI) da UFJF, em atendimento     ao que estabelece a Lei Sinais e a Resolução Consu 13/2015 (UFJF),     com objetivo de contribuir para a avaliação própria do departamento CME.    Foram aplicados um instrumento para discentes e outro para docentes, ambos contendo     15 questões versando sobre as disciplinas na modalidade presencial oferecidas pela UFJF no     referido período, visando, especificamente, coletar impressões sobre: atuação docente, atuação discente,     recursos empregados, qualidade da disciplina ministrada.     As respostas foram colhidas      com participação espontânea e garantia de    sigilo de participantes e avaliados.
\section{QUESTÕES APRESENTADAS NOS FORMULÁRIOS}
As seguintes questões foram {\bf objeto de avaliação pelos discentes} através do SIGA.

\begin{center}
\small{
\begin{tabularx}{\linewidth}{l|X}
Q01&Sua motivação para estudar esta disciplina foi alta.\\\\
Q02&Você teve facilidade para acompanhar as atividades da disciplina.\\\\
Q03&Seu empenho durante a disciplina foi elevado.\\\\
Q04&O plano de ensino apresentado pelo(a) docente contém: ementa, objetivos, metodologias de ensino, critérios de avaliação, conteúdos e bibliografia da disciplina.\\\\
Q05&Os conteúdos trabalhados na disciplina foram coerentes com os que foram apresentados no plano de ensino.\\\\
Q06&O(A) docente deu retorno sobre todas as avaliações realizadas: correções, discussão dos pontos principais, esclarecimentos sobre os erros cometidos.\\\\
Q07&Os recursos didáticos, audiovisuais, tecnológicos empregados pelo(a) docente contribuíram para que a aula fosse mais produtiva.\\\\
Q08&O(A) docente cumpriu a carga horária prevista no semestre para a disciplina.\\\\
Q09&O grau de dificuldade das avaliações foi compatível com as aulas ministradas.\\\\
Q10&A disciplina contribuiu na preparação para o exercício profissional.\\\\
Q11&A disciplina contribuiu para a aquisição de formação teórica/prática na área. \\\\
Q12&A disciplina contribuiu para a aquisição de cultura geral.\\\\
Q13&O(A) docente dispôs de horários de atendimento extra-classe.\\\\
Q14& As notas de todas as avaliações parciais foram divulgadas no SIGA até 3 (três) dias antes da data da avaliação subsequente, conforme determina o RAG. \\\\
Q15&O(A) docente manteve um comportamento respeitoso e cortês em relação aos discentes.
\end{tabularx}
}
\end{center}
As questões abaixo foram {\bf objeto de avaliação pelos docentes} através do SIGA.

\begin{center}
\small{
\begin{tabularx}{\linewidth}{c|X}
Q01&A turma se mostrou motivada para estudar a disciplina.\\\\
Q02&A turma teve facilidade para acompanhar acompanhar as atividades da disciplina\\\\
Q03&A turma se empenhou em acompanhar as atividades da disciplina.\\\\
Q04&Você apresentou plano de ensino contendo: ementa, objetivos, metodologias de ensino, critérios de avaliação, conteúdos e bibliografia da disciplina.\\\\
Q05&Os conteúdos trabalhados na disciplina foram coerentes com os que foram apresentados no plano de ensino.\\\\
Q06&Você deu retorno sobre todas as avaliações realizadas: correções, discussão dos pontos principais, esclarecimentos sobre os erros cometidos.\\\\
Q07&Os recursos didáticos, audiovisuais, tecnológicos empregados pelo(a) docente contribuíram para que a aula fosse mais produtiva.\\\\
Q08&Você cumpriu a carga horária prevista no semestre para a disciplina.\\\\
Q09&O grau de dificuldade das avaliações foi compatível com as aulas ministradas.\\\\
Q10&A disciplina contribuiu na preparação para o exercício profissional.\\\\
Q11&A disciplina contribuiu para a aquisição de formação teórica/prática na área. \\\\
Q12&A disciplina contribuiu para a aquisição de cultura geral.\\\\
Q13&Você dispôs de horário de atendimento extra-classe.\\\\
Q14& As notas de todas as avaliações parciais foram divulgadas no SIGA até 3 (três) dias antes da data da avaliação subsequente, conforme determina o RAG. \\\\
Q15&Você manteve um comportamento respeitoso e cortês em relação aos discentes.
\end{tabularx}
}
\end{center}
\section{RESPOSTAS}
As questões podem ser respondidas com um número de 1 a 5 numa escala desde {\it Discordo Totalmente} (1) a {\it Concordo Totalmente} (5). Não sendo permitidas múltiplas respostas e sendo possível a alteração antes do envio do formulário. O valor 0 (zero) indica {\it Não se Aplica}.

Os nomes das disciplinas foram alterados para garantir a segurança da informações de modo que não seja possível a identificação das turmas, dos alunos  e dos professores que responderam a avaliação.

\begin{figure}[h]
\centering
\includegraphics[width=0.85\linewidth]{analise_geral_departamento_CME_docentes.png}
\includegraphics[width=0.85\linewidth]{analise_geral_departamento_CME_alunos.png}
\caption{\label{fig:analise_geral_departamento}            Panorama geral das respostas de todas as  disciplinas do departamenbto para as questões apresentadas.}
\end{figure}

\begin{figure}[h]
\centering
\includegraphics[width=0.999\linewidth]{matriz_corr__CME_docentes.png}
\caption{\label{fig:corr_docentes}Correlação das respostas dos professores do departamento CME. Entradas com em branco indicam que não havia informação suficiente para o cálculo das correlações.}
\end{figure}

\begin{figure}[h]
\centering
\includegraphics[width=0.999\linewidth]{matriz_corr__CME_alunos.png}
\caption{\label{fig:corr_alunos}Correlação das respostas dos alunos para o departamento CME. Entradas com em branco indicam que não havia informação suficiente para o cálculo das correlações.}
\end{figure}
\begin{figure}[h]
\centering
\includegraphics[width=0.485\linewidth]{analise_disciplina_departamento_CME_605FC9FB5979279D062BBE0BDED565B7_docentes.png}
\includegraphics[width=0.485\linewidth]{analise_disciplina_departamento_CME_605FC9FB5979279D062BBE0BDED565B7_alunos.png}
\caption{\label{fig:analise_geral_departamento}                Distribuição das respostas para a disciplina 605FC9FB5979279D062BBE0BDED565B7. }
\end{figure}
\begin{figure}[h]
\centering
\includegraphics[width=0.485\linewidth]{analise_disciplina_departamento_CME_73773880DCDBA50612EAA9460169AAFE_docentes.png}
\includegraphics[width=0.485\linewidth]{analise_disciplina_departamento_CME_73773880DCDBA50612EAA9460169AAFE_alunos.png}
\caption{\label{fig:analise_geral_departamento}                Distribuição das respostas para a disciplina 73773880DCDBA50612EAA9460169AAFE. }
\end{figure}
\begin{figure}[h]
\centering
\includegraphics[width=0.485\linewidth]{analise_disciplina_departamento_CME_2FFADE175C7491267BAB63921826D825_docentes.png}
\includegraphics[width=0.485\linewidth]{analise_disciplina_departamento_CME_2FFADE175C7491267BAB63921826D825_alunos.png}
\caption{\label{fig:analise_geral_departamento}                Distribuição das respostas para a disciplina 2FFADE175C7491267BAB63921826D825. }
\end{figure}
\begin{figure}[h]
\centering
\includegraphics[width=0.485\linewidth]{analise_disciplina_departamento_CME_8E3408A27A1E58389745DA45F0D2B3E0_docentes.png}
\includegraphics[width=0.485\linewidth]{analise_disciplina_departamento_CME_8E3408A27A1E58389745DA45F0D2B3E0_alunos.png}
\caption{\label{fig:analise_geral_departamento}                Distribuição das respostas para a disciplina 8E3408A27A1E58389745DA45F0D2B3E0. }
\end{figure}
\begin{figure}[h]
\centering
\includegraphics[width=0.485\linewidth]{analise_disciplina_departamento_CME_3D354D195481BB9D5F88EE842C8B7620_docentes.png}
\caption{\label{fig:analise_geral_departamento}                Distribuição das respostas para a disciplina 3D354D195481BB9D5F88EE842C8B7620. O gráfico da avaliação dos alunos não foi mostrado  por um dos motivos:  a disciplina não foi avaliada pelos alunos ou somente um aluno realizou a avaliação. }
\end{figure}
\begin{figure}[h]
\centering
\includegraphics[width=0.485\linewidth]{analise_disciplina_departamento_CME_03FC3A10690676E264BD9514DCAD092A_docentes.png}
\includegraphics[width=0.485\linewidth]{analise_disciplina_departamento_CME_03FC3A10690676E264BD9514DCAD092A_alunos.png}
\caption{\label{fig:analise_geral_departamento}                Distribuição das respostas para a disciplina 03FC3A10690676E264BD9514DCAD092A. }
\end{figure}
\begin{figure}[h]
\centering
\includegraphics[width=0.485\linewidth]{analise_disciplina_departamento_CME_065F98D1856F9377D027EF81273F0197_docentes.png}
\includegraphics[width=0.485\linewidth]{analise_disciplina_departamento_CME_065F98D1856F9377D027EF81273F0197_alunos.png}
\caption{\label{fig:analise_geral_departamento}                Distribuição das respostas para a disciplina 065F98D1856F9377D027EF81273F0197. }
\end{figure}
\begin{figure}[h]
\centering
\includegraphics[width=0.485\linewidth]{analise_disciplina_departamento_CME_C222E27D4EA12E2B126E4008F883B752_docentes.png}
\caption{\label{fig:analise_geral_departamento}                Distribuição das respostas para a disciplina C222E27D4EA12E2B126E4008F883B752. O gráfico da avaliação dos alunos não foi mostrado  por um dos motivos:  a disciplina não foi avaliada pelos alunos ou somente um aluno realizou a avaliação. }
\end{figure}
\begin{figure}[h]
\centering
\includegraphics[width=0.485\linewidth]{analise_disciplina_departamento_CME_69EE553B4824C87703B93794083E5F96_docentes.png}
\includegraphics[width=0.485\linewidth]{analise_disciplina_departamento_CME_69EE553B4824C87703B93794083E5F96_alunos.png}
\caption{\label{fig:analise_geral_departamento}                Distribuição das respostas para a disciplina 69EE553B4824C87703B93794083E5F96. }
\end{figure}
\begin{figure}[h]
\centering
\includegraphics[width=0.485\linewidth]{analise_disciplina_departamento_CME_209CDEE7A782E22D96B8DB983D820E22_docentes.png}
\includegraphics[width=0.485\linewidth]{analise_disciplina_departamento_CME_209CDEE7A782E22D96B8DB983D820E22_alunos.png}
\caption{\label{fig:analise_geral_departamento}                Distribuição das respostas para a disciplina 209CDEE7A782E22D96B8DB983D820E22. }
\end{figure}
\begin{figure}[h]
\centering
\includegraphics[width=0.485\linewidth]{analise_disciplina_departamento_CME_9ABA617D722D54E0EA431FECF1524E85_docentes.png}
\includegraphics[width=0.485\linewidth]{analise_disciplina_departamento_CME_9ABA617D722D54E0EA431FECF1524E85_alunos.png}
\caption{\label{fig:analise_geral_departamento}                Distribuição das respostas para a disciplina 9ABA617D722D54E0EA431FECF1524E85. }
\end{figure}
\begin{figure}[h]
\centering
\includegraphics[width=0.485\linewidth]{analise_disciplina_departamento_CME_9330579B3E292876C86E9862EAA8769C_docentes.png}
\includegraphics[width=0.485\linewidth]{analise_disciplina_departamento_CME_9330579B3E292876C86E9862EAA8769C_alunos.png}
\caption{\label{fig:analise_geral_departamento}                Distribuição das respostas para a disciplina 9330579B3E292876C86E9862EAA8769C. }
\end{figure}
\begin{figure}[h]
\centering
\includegraphics[width=0.485\linewidth]{analise_disciplina_departamento_CME_A82FEF00D7E151CC111BF9A78F2E7B52_docentes.png}
\includegraphics[width=0.485\linewidth]{analise_disciplina_departamento_CME_A82FEF00D7E151CC111BF9A78F2E7B52_alunos.png}
\caption{\label{fig:analise_geral_departamento}                Distribuição das respostas para a disciplina A82FEF00D7E151CC111BF9A78F2E7B52. }
\end{figure}
\begin{figure}[h]
\centering
\includegraphics[width=0.485\linewidth]{analise_disciplina_departamento_CME_6D73686C869050ECA4F9D4B342A6D53B_docentes.png}
\includegraphics[width=0.485\linewidth]{analise_disciplina_departamento_CME_6D73686C869050ECA4F9D4B342A6D53B_alunos.png}
\caption{\label{fig:analise_geral_departamento}                Distribuição das respostas para a disciplina 6D73686C869050ECA4F9D4B342A6D53B. }
\end{figure}
\begin{figure}[h]
\centering
\includegraphics[width=0.485\linewidth]{analise_disciplina_departamento_CME_E0121DC581E973C3EB266F92AD7091C1_docentes.png}
\includegraphics[width=0.485\linewidth]{analise_disciplina_departamento_CME_E0121DC581E973C3EB266F92AD7091C1_alunos.png}
\caption{\label{fig:analise_geral_departamento}                Distribuição das respostas para a disciplina E0121DC581E973C3EB266F92AD7091C1. }
\end{figure}
\begin{figure}[h]
\centering
\includegraphics[width=0.485\linewidth]{analise_disciplina_departamento_CME_841E0F1079E621C7F4E69FD44B45217E_docentes.png}
\includegraphics[width=0.485\linewidth]{analise_disciplina_departamento_CME_841E0F1079E621C7F4E69FD44B45217E_alunos.png}
\caption{\label{fig:analise_geral_departamento}                Distribuição das respostas para a disciplina 841E0F1079E621C7F4E69FD44B45217E. }
\end{figure}

\end{document}

\documentclass[a4paper,10pt]{article}
\usepackage{ucs}
\usepackage[utf8]{inputenc}
\usepackage[brazil]{babel}
\usepackage{fontenc}
\usepackage{indentfirst}
\usepackage{graphicx,tabularx}
\usepackage[]{hyperref}
\sloppy
\date{Data de processamento: \today}\begin{document}
\author{Diretoria de Avaliação Institucional (DIAVI) \\ Universidade Federal de Juiz de Fora}

\title{RELATÓRIO DE RESULTADOS DA AVALIAÇÃO DO CURSO DE LETRAS}
\maketitle
\section{INTRODUÇÃO}
Este relatório visa apresentar os resultados da avaliação de disciplinas do Curso     de LETRAS da Universidade Federal de Juiz de Fora, código e-MEC 13101, realizada pela     Diretoria de Avaliação Institucional e os encaminhamentos propostos a     partir destes resultados.

\begin{center}
\begin{tabularx}{\linewidth}{r|X}

Público-alvo:& Curso de LETRAS\\

Período de coleta de dados:& 2019/3 \\

Forma de aplicação:& Online, por meio do SIGA\\

Alunos respondentes:& 1\\

Professores avaliados:& 8\\

Professores respondentes:& 4\\
\end{tabularx}
\end{center}
 Para resguardar o sigilo dos participantes, os resultados relativos aos discentes serão omitidos na ausência de alunos respondentes ou na eventualidade de somente um aluno responder o questionário.
\section{MÉTODOS}
Este relatório se refere ao período 2019/3, com base em dados     coletados através da aplicação de instrumentos de avaliação via SIGA     implementados pela Diretoria de Avaliação Institucional (DIAVI) da UFJF, em atendimento     ao que estabelece a Lei Sinais e a Resolução Consu 13/2015 (UFJF),     com objetivo de contribuir para a avaliação própria do curso de LETRAS (código e-MEC13101). Foram aplicados um instrumento para discentes e outro para docentes, ambos contendo     15 questões versando sobre as disciplinas na modalidade presencial oferecidas pela UFJF no     referido período, visando, especificamente, coletar impressões sobre: atuação docente, atuação discente,     recursos empregados, qualidade da disciplina ministrada.     As respostas foram colhidas entre os dias 19/07/2018 e 12/08/2018, com participação espontânea e garantia de    sigilo de participantes e avaliados.
\section{QUESTÕES APRESENTADAS NO FORMULÁRIO}
{ \it O questionário discente foi omitido pois ocorreu uma das condições listadas a seguir: ausência de alunos respondentes, ou somente um aluno responder o questionário.}

As questões abaixo foram {\bf objeto de avaliação pelos docentes} através do SIGA.

\small{
\begin{center}
\begin{tabularx}{\linewidth}{c|X}
Q01&A turma se mostrou motivada para estudar a disciplina.\\\\
Q02&A turma teve facilidade para acompanhar acompanhar as atividades da disciplina\\\\
Q03&A turma se empenhou em acompanhar as atividades da disciplina.\\\\
Q04&Você apresentou plano de ensino contendo: ementa, objetivos, metodologias de ensino, critérios de avaliação, conteúdos e bibliografia da disciplina.\\\\
Q05&Os conteúdos trabalhados na disciplina foram coerentes com os que foram apresentados no plano de ensino.\\\\
Q06&Você deu retorno sobre todas as avaliações realizadas: correções, discussão dos pontos principais, esclarecimentos sobre os erros cometidos.\\\\
Q07&Os recursos didáticos, audiovisuais, tecnológicos empregados pelo(a) docente contribuíram para que a aula fosse mais produtiva.\\\\
Q08&Você cumpriu a carga horária prevista no semestre para a disciplina.\\\\
Q09&O grau de dificuldade das avaliações foi compatível com as aulas ministradas.\\\\
Q10&A disciplina contribuiu na preparação para o exercício profissional.\\\\
Q11&A disciplina contribuiu para a aquisição de formação teórica/prática na área. \\\\
Q12&A disciplina contribuiu para a aquisição de cultura geral.\\\\
Q13&Você dispôs de horário de atendimento extra-classe.\\\\
Q14& As notas de todas as avaliações parciais foram divulgadas no SIGA até 3 (três) dias antes da data da avaliação subsequente, conforme determina o RAG. \\\\
Q15&Você manteve um comportamento respeitoso e cortês em relação aos discentes.
\end{tabularx}
\end{center}
}
\section{RESPOSTAS}
As questões podem ser respondidas com um número de 1 a 5 numa escala que vai de {\it Discordo Totalmente} (1) a {\it Concordo Totalmente} (5). Não sendo permitidas múltiplas respostas e sendo possível a alteração antes do envio do formulário. O valor 0 (zero) indica {\it Não se Aplica}.


\begin{figure}[h]
\centering
\includegraphics[width=0.999\linewidth]{resposta_docentes_questoes_curso_13101.png}
\caption{\label{fig:resposta_questoes_curso}Distribuição das respostas dos alunos e docentes para as questões apresentadas. O gráfico da avaliação dos alunos não foi mostrado  por um dos motivos:  não houve avaliação por parte dos alunos ou somente um aluno realizou a avaliação. }
\end{figure}

\begin{figure}[h]
\centering
\includegraphics[width=0.999\linewidth]{matriz_corr__docentes_13101.png}
\caption{\label{fig:corr_docentes}Correlação das respostas dos docentes do curso LETRAS para as questões apresentadas. . Entradas com em branco indicam que não havia informação suficiente para o cálculo das correlações.}
\end{figure}

\end{document}
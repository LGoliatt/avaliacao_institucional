
\documentclass[a4paper,10pt]{article}
\usepackage{ucs}
\usepackage[utf8]{inputenc}
\usepackage[brazil]{babel}
\usepackage{fontenc}
\usepackage{indentfirst}
\usepackage{graphicx,tabularx}
\usepackage[]{hyperref}
\sloppy
\date{Data de processamento: \today}\begin{document}
\author{Diretoria de Avaliação Institucional (DIAVI) \\ Universidade Federal de Juiz de Fora}

\title{RELATÓRIO DE RESULTADOS DA AVALIAÇÃO DO CURSO DE MEDICINA}
\maketitle
\section{INTRODUÇÃO}
Este relatório visa apresentar os resultados da avaliação de disciplinas do Curso     de MEDICINA da Universidade Federal de Juiz de Fora, código e-MEC 5001167, realizada pela     Diretoria de Avaliação Institucional e os encaminhamentos propostos a     partir destes resultados.

\begin{center}
\begin{tabularx}{\linewidth}{r|X}

Público-alvo:& Curso de MEDICINA\\

Período de coleta de dados:& 2019/3 \\

Forma de aplicação:& Online, por meio do SIGA\\

Alunos respondentes:& 74\\

Professores avaliados:& 91\\

Professores respondentes:& 33\\
\end{tabularx}
\end{center}

\section{MÉTODOS}
Este relatório se refere ao período 2019/3, com base em dados     coletados através da aplicação de instrumentos de avaliação via SIGA     implementados pela Diretoria de Avaliação Institucional (DIAVI) da UFJF, em atendimento     ao que estabelece a Lei Sinais e a Resolução Consu 13/2015 (UFJF),     com objetivo de contribuir para a avaliação própria do curso de MEDICINA (código e-MEC5001167). Foram aplicados um instrumento para discentes e outro para docentes, ambos contendo     15 questões versando sobre as disciplinas na modalidade presencial oferecidas pela UFJF no     referido período, visando, especificamente, coletar impressões sobre: atuação docente, atuação discente,     recursos empregados, qualidade da disciplina ministrada.     As respostas foram colhidas entre os dias 19/07/2018 e 12/08/2018, com participação espontânea e garantia de    sigilo de participantes e avaliados.
\section{QUESTÕES APRESENTADAS NO FORMULÁRIO}
As seguintes questões foram {\bf objeto de avaliação pelos discentes} através do SIGA.

\begin{center}
\small{
\begin{tabularx}{\linewidth}{l|X}
Q01&Sua motivação para estudar esta disciplina foi alta.\\\\
Q02&Você teve facilidade para acompanhar as atividades da disciplina.\\\\
Q03&Seu empenho durante a disciplina foi elevado.\\\\
Q04&O plano de ensino apresentado pelo(a) docente contém: ementa, objetivos, metodologias de ensino, critérios de avaliação, conteúdos e bibliografia da disciplina.\\\\
Q05&Os conteúdos trabalhados na disciplina foram coerentes com os que foram apresentados no plano de ensino.\\\\
Q06&O(A) docente deu retorno sobre todas as avaliações realizadas: correções, discussão dos pontos principais, esclarecimentos sobre os erros cometidos.\\\\
Q07&Os recursos didáticos, audiovisuais, tecnológicos empregados pelo(a) docente contribuíram para que a aula fosse mais produtiva.\\\\
Q08&O(A) docente cumpriu a carga horária prevista no semestre para a disciplina.\\\\
Q09&O grau de dificuldade das avaliações foi compatível com as aulas ministradas.\\\\
Q10&A disciplina contribuiu na preparação para o exercício profissional.\\\\
Q11&A disciplina contribuiu para a aquisição de formação teórica/prática na área. \\\\
Q12&A disciplina contribuiu para a aquisição de cultura geral.\\\\
Q13&O(A) docente dispôs de horários de atendimento extra-classe.\\\\
Q14& As notas de todas as avaliações parciais foram divulgadas no SIGA até 3 (três) dias antes da data da avaliação subsequente, conforme determina o RAG. \\\\
Q15&O(A) docente manteve um comportamento respeitoso e cortês em relação aos discentes.
\end{tabularx}
}
\end{center}
As questões abaixo foram {\bf objeto de avaliação pelos docentes} através do SIGA.

\small{
\begin{center}
\begin{tabularx}{\linewidth}{c|X}
Q01&A turma se mostrou motivada para estudar a disciplina.\\\\
Q02&A turma teve facilidade para acompanhar acompanhar as atividades da disciplina\\\\
Q03&A turma se empenhou em acompanhar as atividades da disciplina.\\\\
Q04&Você apresentou plano de ensino contendo: ementa, objetivos, metodologias de ensino, critérios de avaliação, conteúdos e bibliografia da disciplina.\\\\
Q05&Os conteúdos trabalhados na disciplina foram coerentes com os que foram apresentados no plano de ensino.\\\\
Q06&Você deu retorno sobre todas as avaliações realizadas: correções, discussão dos pontos principais, esclarecimentos sobre os erros cometidos.\\\\
Q07&Os recursos didáticos, audiovisuais, tecnológicos empregados pelo(a) docente contribuíram para que a aula fosse mais produtiva.\\\\
Q08&Você cumpriu a carga horária prevista no semestre para a disciplina.\\\\
Q09&O grau de dificuldade das avaliações foi compatível com as aulas ministradas.\\\\
Q10&A disciplina contribuiu na preparação para o exercício profissional.\\\\
Q11&A disciplina contribuiu para a aquisição de formação teórica/prática na área. \\\\
Q12&A disciplina contribuiu para a aquisição de cultura geral.\\\\
Q13&Você dispôs de horário de atendimento extra-classe.\\\\
Q14& As notas de todas as avaliações parciais foram divulgadas no SIGA até 3 (três) dias antes da data da avaliação subsequente, conforme determina o RAG. \\\\
Q15&Você manteve um comportamento respeitoso e cortês em relação aos discentes.
\end{tabularx}
\end{center}
}
\section{RESPOSTAS}
As questões podem ser respondidas com um número de 1 a 5 numa escala que vai de {\it Discordo Totalmente} (1) a {\it Concordo Totalmente} (5). Não sendo permitidas múltiplas respostas e sendo possível a alteração antes do envio do formulário. O valor 0 (zero) indica {\it Não se Aplica}.


\begin{figure}[h]
\centering
\includegraphics[width=0.999\linewidth]{resposta_alunos_questoes_curso_5001167.png}
\includegraphics[width=0.999\linewidth]{resposta_docentes_questoes_curso_5001167.png}
\caption{\label{fig:resposta_questoes_curso}Distribuição das respostas dos alunos e docentes para as questões apresentadas. }
\end{figure}

\begin{figure}[h]
\centering
\includegraphics[width=0.999\linewidth]{matriz_corr__alunos_5001167.png}
\caption{\label{fig:corr_alunos}Correlação das respostas dos alunos do curso MEDICINA para as questões apresentadas. . Entradas com em branco indicam que não havia informação suficiente para o cálculo das correlações.}
\end{figure}

\begin{figure}[h]
\centering
\includegraphics[width=0.999\linewidth]{matriz_corr__docentes_5001167.png}
\caption{\label{fig:corr_docentes}Correlação das respostas dos docentes do curso MEDICINA para as questões apresentadas. . Entradas com em branco indicam que não havia informação suficiente para o cálculo das correlações.}
\end{figure}
\begin{figure}[h]
\centering
\includegraphics[width=0.85\linewidth]{ingresso_discentes_curso_ano_5001167.png}
\caption{\label{fig:ingresso_ano} Perfil do ano de ingresso dos alunos respondentes em função do ano ingresso.}
\end{figure}

\begin{figure}[h]
\centering
\includegraphics[width=0.85\linewidth]{quantitativos_estado_de_origem_5001167.png}
\caption{\label{fig:estado_ano} Perfil do estado de origem dos alunos respondentes em função do ano de ingresso.}
\end{figure}

\begin{figure}[h]
\centering
\includegraphics[width=0.85\linewidth]{quantitativos_bolsa_de_apoio_5001167.png}
\caption{\label{fig:bolsa_ano} Perfil das bolsas de apoio  dos alunos respondentes em função do ano de ingresso.}
\end{figure}


\end{document}
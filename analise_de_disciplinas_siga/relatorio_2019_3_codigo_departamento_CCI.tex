
\documentclass[a4paper,10pt]{article}
\usepackage{ucs}
\usepackage[utf8]{inputenc}
\usepackage[brazil]{babel}
\usepackage{a4wide}
\usepackage{fontenc}
\usepackage{times}
\usepackage{indentfirst}
\usepackage{graphicx,tabularx}
\usepackage[]{hyperref}
\sloppy
\date{Data de processamento: \today}\begin{document}
\author{Diretoria de Avaliação Institucional (DIAVI) \\ Universidade Federal de Juiz de Fora}

\title{RELATÓRIO DE RESULTADOS DA AVALIAÇÃO DE DEPARTAMENTO\\ Código do Departamento: CCI}
\maketitle
\section{INTRODUÇÃO}
Este relatório objetiva apresentar os resultados da avaliação de disciplinas do Departamento     de código CCI da Universidade Federal de Juiz de Fora, realizada pela     Diretoria de Avaliação Institucional e os encaminhamentos propostos a     partir destes resultados.

\begin{center}
\begin{tabularx}{\linewidth}{X|l}

Público-alvo:& Departamento  CCI\\

Período de coleta de dados:& 2019/3.\\

Forma de aplicação:& Online, por meio do SIGA.\\

Docentes respondentes:& 7\\

Número de disciplinas do departamento avaliadas pelos Docentes:& 15\\

Alunos   respondentes:& 112\\

Número de disciplinas do departamento  avaliadas pelos   Alunos:& 13\\
\end{tabularx}
\end{center}

Os resultados relativos as respostas dos discentes serão omitidos na ausência de alunos respondentes, ou na eventualidade de somente um aluno responder o questionário discente.
\section{MÉTODOS}
Este relatório se refere ao período 2019/3, com base em dados     coletados através da aplicação de instrumentos de avaliação via SIGA     implementados pela Diretoria de Avaliação Institucional (DIAVI) da UFJF, em atendimento     ao que estabelece a Lei Sinais e a Resolução Consu 13/2015 (UFJF),     com objetivo de contribuir para a avaliação própria do departamento CCI.    Foram aplicados um instrumento para discentes e outro para docentes, ambos contendo     15 questões versando sobre as disciplinas na modalidade presencial oferecidas pela UFJF no     referido período, visando, especificamente, coletar impressões sobre: atuação docente, atuação discente,     recursos empregados, qualidade da disciplina ministrada.     As respostas foram colhidas      com participação espontânea e garantia de    sigilo de participantes e avaliados.
\section{QUESTÕES APRESENTADAS NOS FORMULÁRIOS}
As seguintes questões foram {\bf objeto de avaliação pelos discentes} através do SIGA.

\begin{center}
\small{
\begin{tabularx}{\linewidth}{l|X}
Q01&Sua motivação para estudar esta disciplina foi alta.\\\\
Q02&Você teve facilidade para acompanhar as atividades da disciplina.\\\\
Q03&Seu empenho durante a disciplina foi elevado.\\\\
Q04&O plano de ensino apresentado pelo(a) docente contém: ementa, objetivos, metodologias de ensino, critérios de avaliação, conteúdos e bibliografia da disciplina.\\\\
Q05&Os conteúdos trabalhados na disciplina foram coerentes com os que foram apresentados no plano de ensino.\\\\
Q06&O(A) docente deu retorno sobre todas as avaliações realizadas: correções, discussão dos pontos principais, esclarecimentos sobre os erros cometidos.\\\\
Q07&Os recursos didáticos, audiovisuais, tecnológicos empregados pelo(a) docente contribuíram para que a aula fosse mais produtiva.\\\\
Q08&O(A) docente cumpriu a carga horária prevista no semestre para a disciplina.\\\\
Q09&O grau de dificuldade das avaliações foi compatível com as aulas ministradas.\\\\
Q10&A disciplina contribuiu na preparação para o exercício profissional.\\\\
Q11&A disciplina contribuiu para a aquisição de formação teórica/prática na área. \\\\
Q12&A disciplina contribuiu para a aquisição de cultura geral.\\\\
Q13&O(A) docente dispôs de horários de atendimento extra-classe.\\\\
Q14& As notas de todas as avaliações parciais foram divulgadas no SIGA até 3 (três) dias antes da data da avaliação subsequente, conforme determina o RAG. \\\\
Q15&O(A) docente manteve um comportamento respeitoso e cortês em relação aos discentes.
\end{tabularx}
}
\end{center}
As questões abaixo foram {\bf objeto de avaliação pelos docentes} através do SIGA.

\begin{center}
\small{
\begin{tabularx}{\linewidth}{c|X}
Q01&A turma se mostrou motivada para estudar a disciplina.\\\\
Q02&A turma teve facilidade para acompanhar acompanhar as atividades da disciplina\\\\
Q03&A turma se empenhou em acompanhar as atividades da disciplina.\\\\
Q04&Você apresentou plano de ensino contendo: ementa, objetivos, metodologias de ensino, critérios de avaliação, conteúdos e bibliografia da disciplina.\\\\
Q05&Os conteúdos trabalhados na disciplina foram coerentes com os que foram apresentados no plano de ensino.\\\\
Q06&Você deu retorno sobre todas as avaliações realizadas: correções, discussão dos pontos principais, esclarecimentos sobre os erros cometidos.\\\\
Q07&Os recursos didáticos, audiovisuais, tecnológicos empregados pelo(a) docente contribuíram para que a aula fosse mais produtiva.\\\\
Q08&Você cumpriu a carga horária prevista no semestre para a disciplina.\\\\
Q09&O grau de dificuldade das avaliações foi compatível com as aulas ministradas.\\\\
Q10&A disciplina contribuiu na preparação para o exercício profissional.\\\\
Q11&A disciplina contribuiu para a aquisição de formação teórica/prática na área. \\\\
Q12&A disciplina contribuiu para a aquisição de cultura geral.\\\\
Q13&Você dispôs de horário de atendimento extra-classe.\\\\
Q14& As notas de todas as avaliações parciais foram divulgadas no SIGA até 3 (três) dias antes da data da avaliação subsequente, conforme determina o RAG. \\\\
Q15&Você manteve um comportamento respeitoso e cortês em relação aos discentes.
\end{tabularx}
}
\end{center}
\section{RESPOSTAS}
As questões podem ser respondidas com um número de 1 a 5 numa escala desde {\it Discordo Totalmente} (1) a {\it Concordo Totalmente} (5). Não sendo permitidas múltiplas respostas e sendo possível a alteração antes do envio do formulário. O valor 0 (zero) indica {\it Não se Aplica}.

Os nomes das disciplinas foram alterados para garantir a segurança da informações de modo que não seja possível a identificação das turmas, dos alunos  e dos professores que responderam a avaliação.

\begin{figure}[h]
\centering
\includegraphics[width=0.85\linewidth]{analise_geral_departamento_CCI_docentes.png}
\includegraphics[width=0.85\linewidth]{analise_geral_departamento_CCI_alunos.png}
\caption{\label{fig:analise_geral_departamento}            Panorama geral das respostas de todas as  disciplinas do departamenbto para as questões apresentadas.}
\end{figure}

\begin{figure}[h]
\centering
\includegraphics[width=0.999\linewidth]{matriz_corr__CCI_docentes.png}
\caption{\label{fig:corr_docentes}Correlação das respostas dos professores do departamento CCI. Entradas com em branco indicam que não havia informação suficiente para o cálculo das correlações.}
\end{figure}

\begin{figure}[h]
\centering
\includegraphics[width=0.999\linewidth]{matriz_corr__CCI_alunos.png}
\caption{\label{fig:corr_alunos}Correlação das respostas dos alunos para o departamento CCI. Entradas com em branco indicam que não havia informação suficiente para o cálculo das correlações.}
\end{figure}
\begin{figure}[h]
\centering
\includegraphics[width=0.485\linewidth]{analise_disciplina_departamento_CCI_B75E63024E6E61CAA28F3900C336EA77_docentes.png}
\includegraphics[width=0.485\linewidth]{analise_disciplina_departamento_CCI_B75E63024E6E61CAA28F3900C336EA77_alunos.png}
\caption{\label{fig:analise_geral_departamento}                Distribuição das respostas para a disciplina B75E63024E6E61CAA28F3900C336EA77. }
\end{figure}
\begin{figure}[h]
\centering
\includegraphics[width=0.485\linewidth]{analise_disciplina_departamento_CCI_3480D53143CFBC36850014DFD7403886_docentes.png}
\includegraphics[width=0.485\linewidth]{analise_disciplina_departamento_CCI_3480D53143CFBC36850014DFD7403886_alunos.png}
\caption{\label{fig:analise_geral_departamento}                Distribuição das respostas para a disciplina 3480D53143CFBC36850014DFD7403886. }
\end{figure}
\begin{figure}[h]
\centering
\includegraphics[width=0.485\linewidth]{analise_disciplina_departamento_CCI_DD8C73901C2A8043822C5FC9353E19B7_docentes.png}
\includegraphics[width=0.485\linewidth]{analise_disciplina_departamento_CCI_DD8C73901C2A8043822C5FC9353E19B7_alunos.png}
\caption{\label{fig:analise_geral_departamento}                Distribuição das respostas para a disciplina DD8C73901C2A8043822C5FC9353E19B7. }
\end{figure}
\begin{figure}[h]
\centering
\includegraphics[width=0.485\linewidth]{analise_disciplina_departamento_CCI_A7838FC1DE3CD5F57EE524985D215FFF_docentes.png}
\includegraphics[width=0.485\linewidth]{analise_disciplina_departamento_CCI_A7838FC1DE3CD5F57EE524985D215FFF_alunos.png}
\caption{\label{fig:analise_geral_departamento}                Distribuição das respostas para a disciplina A7838FC1DE3CD5F57EE524985D215FFF. }
\end{figure}
\begin{figure}[h]
\centering
\includegraphics[width=0.485\linewidth]{analise_disciplina_departamento_CCI_E5B0181C0049AC53101CC18E149103F6_docentes.png}
\includegraphics[width=0.485\linewidth]{analise_disciplina_departamento_CCI_E5B0181C0049AC53101CC18E149103F6_alunos.png}
\caption{\label{fig:analise_geral_departamento}                Distribuição das respostas para a disciplina E5B0181C0049AC53101CC18E149103F6. }
\end{figure}
\begin{figure}[h]
\centering
\includegraphics[width=0.485\linewidth]{analise_disciplina_departamento_CCI_B09B4CEC941B3480AE64C8D29418C915_docentes.png}
\includegraphics[width=0.485\linewidth]{analise_disciplina_departamento_CCI_B09B4CEC941B3480AE64C8D29418C915_alunos.png}
\caption{\label{fig:analise_geral_departamento}                Distribuição das respostas para a disciplina B09B4CEC941B3480AE64C8D29418C915. }
\end{figure}
\begin{figure}[h]
\centering
\includegraphics[width=0.485\linewidth]{analise_disciplina_departamento_CCI_1FC5DD18056D20BF9E88ACD5FE0BFE15_docentes.png}
\includegraphics[width=0.485\linewidth]{analise_disciplina_departamento_CCI_1FC5DD18056D20BF9E88ACD5FE0BFE15_alunos.png}
\caption{\label{fig:analise_geral_departamento}                Distribuição das respostas para a disciplina 1FC5DD18056D20BF9E88ACD5FE0BFE15. }
\end{figure}
\begin{figure}[h]
\centering
\includegraphics[width=0.485\linewidth]{analise_disciplina_departamento_CCI_6F681CF6709C5AC5E84A99CCA67DED2E_docentes.png}
\includegraphics[width=0.485\linewidth]{analise_disciplina_departamento_CCI_6F681CF6709C5AC5E84A99CCA67DED2E_alunos.png}
\caption{\label{fig:analise_geral_departamento}                Distribuição das respostas para a disciplina 6F681CF6709C5AC5E84A99CCA67DED2E. }
\end{figure}
\begin{figure}[h]
\centering
\includegraphics[width=0.485\linewidth]{analise_disciplina_departamento_CCI_15B356032DF2B53C0B22AB222BFF3E50_docentes.png}
\includegraphics[width=0.485\linewidth]{analise_disciplina_departamento_CCI_15B356032DF2B53C0B22AB222BFF3E50_alunos.png}
\caption{\label{fig:analise_geral_departamento}                Distribuição das respostas para a disciplina 15B356032DF2B53C0B22AB222BFF3E50. }
\end{figure}
\begin{figure}[h]
\centering
\includegraphics[width=0.485\linewidth]{analise_disciplina_departamento_CCI_A93EEAA6D583861090B3EDFE35C8653F_docentes.png}
\caption{\label{fig:analise_geral_departamento}                Distribuição das respostas para a disciplina A93EEAA6D583861090B3EDFE35C8653F. O gráfico da avaliação dos alunos não foi mostrado  por um dos motivos:  a disciplina não foi avaliada pelos alunos ou somente um aluno realizou a avaliação. }
\end{figure}
\begin{figure}[h]
\centering
\includegraphics[width=0.485\linewidth]{analise_disciplina_departamento_CCI_2EBD2DB5D4B18C014CF4B2E7374C6183_docentes.png}
\caption{\label{fig:analise_geral_departamento}                Distribuição das respostas para a disciplina 2EBD2DB5D4B18C014CF4B2E7374C6183. O gráfico da avaliação dos alunos não foi mostrado  por um dos motivos:  a disciplina não foi avaliada pelos alunos ou somente um aluno realizou a avaliação. }
\end{figure}
\begin{figure}[h]
\centering
\includegraphics[width=0.485\linewidth]{analise_disciplina_departamento_CCI_863959AC482CC3B1088B1C1245A82969_docentes.png}
\includegraphics[width=0.485\linewidth]{analise_disciplina_departamento_CCI_863959AC482CC3B1088B1C1245A82969_alunos.png}
\caption{\label{fig:analise_geral_departamento}                Distribuição das respostas para a disciplina 863959AC482CC3B1088B1C1245A82969. }
\end{figure}
\begin{figure}[h]
\centering
\includegraphics[width=0.485\linewidth]{analise_disciplina_departamento_CCI_E80C9B17E750B9A67225D4E7F5F5C7F7_docentes.png}
\includegraphics[width=0.485\linewidth]{analise_disciplina_departamento_CCI_E80C9B17E750B9A67225D4E7F5F5C7F7_alunos.png}
\caption{\label{fig:analise_geral_departamento}                Distribuição das respostas para a disciplina E80C9B17E750B9A67225D4E7F5F5C7F7. }
\end{figure}
\begin{figure}[h]
\centering
\includegraphics[width=0.485\linewidth]{analise_disciplina_departamento_CCI_0B4DA882B6D8CB8084093E7AF1FA2D9E_docentes.png}
\includegraphics[width=0.485\linewidth]{analise_disciplina_departamento_CCI_0B4DA882B6D8CB8084093E7AF1FA2D9E_alunos.png}
\caption{\label{fig:analise_geral_departamento}                Distribuição das respostas para a disciplina 0B4DA882B6D8CB8084093E7AF1FA2D9E. }
\end{figure}
\begin{figure}[h]
\centering
\includegraphics[width=0.485\linewidth]{analise_disciplina_departamento_CCI_6CD66DB44706CC60232749409676D3AF_docentes.png}
\includegraphics[width=0.485\linewidth]{analise_disciplina_departamento_CCI_6CD66DB44706CC60232749409676D3AF_alunos.png}
\caption{\label{fig:analise_geral_departamento}                Distribuição das respostas para a disciplina 6CD66DB44706CC60232749409676D3AF. }
\end{figure}

\end{document}

\documentclass[a4paper,10pt]{article}
\usepackage{ucs}
\usepackage[utf8]{inputenc}
\usepackage[brazil]{babel}
\usepackage{a4wide}
\usepackage{fontenc}
\usepackage{times}
\usepackage{indentfirst}
\usepackage{graphicx,tabularx}
\usepackage[]{hyperref}
\sloppy
\date{Data de processamento: \today}\begin{document}
\author{Diretoria de Avaliação Institucional (DIAVI) \\ Universidade Federal de Juiz de Fora}

\title{RELATÓRIO DE RESULTADOS DA AVALIAÇÃO DE DEPARTAMENTO\\ Código do Departamento: ANA}
\maketitle
\section{INTRODUÇÃO}
Este relatório objetiva apresentar os resultados da avaliação de disciplinas do Departamento     de código ANA da Universidade Federal de Juiz de Fora, realizada pela     Diretoria de Avaliação Institucional e os encaminhamentos propostos a     partir destes resultados.

\begin{center}
\begin{tabularx}{\linewidth}{X|l}

Público-alvo:& Departamento  ANA\\

Período de coleta de dados:& 2019/3.\\

Forma de aplicação:& Online, por meio do SIGA.\\

Docentes respondentes:& 4\\

Número de disciplinas do departamento avaliadas pelos Docentes:& 11\\

Alunos   respondentes:& 0\\

Número de disciplinas do departamento  avaliadas pelos   Alunos:& 0\\
\end{tabularx}
\end{center}

Os resultados relativos as respostas dos discentes serão omitidos na ausência de alunos respondentes, ou na eventualidade de somente um aluno responder ao questionário discente.
\section{MÉTODOS}
Este relatório se refere ao período 2019/3, com base em dados     coletados através da aplicação de instrumentos de avaliação via SIGA     implementados pela Diretoria de Avaliação Institucional (DIAVI) da UFJF, em atendimento     ao que estabelece a Lei Sinais e a Resolução Consu 13/2015 (UFJF),     com objetivo de contribuir para a avaliação própria do departamento ANA.    Foram aplicados um instrumento para discentes e outro para docentes, ambos contendo     15 questões versando sobre as disciplinas na modalidade presencial oferecidas pela UFJF no     referido período, visando, especificamente, coletar impressões sobre: atuação docente, atuação discente,     recursos empregados, qualidade da disciplina ministrada.     As respostas foram colhidas      com participação espontânea e garantia de    sigilo de participantes e avaliados. Para resguardar o sigilo dos participantes, os resultados relativos aos discentes serão omitidos na ausência de alunos respondentes, ou na eventualidade de somente um aluno responder ao questionário.
\section{QUESTÕES APRESENTADAS NOS FORMULÁRIOS}
{ \it O questionário discente foi omitido pois ocorreu uma das condições listadas a seguir: ausência de alunos respondentes, ou somente um aluno responder ao questionário.}

As questões abaixo foram {\bf objeto de avaliação pelos docentes} através do SIGA.

\begin{center}
\small{
\begin{tabularx}{\linewidth}{c|X}
Q01&A turma se mostrou motivada para estudar a disciplina.\\\\
Q02&A turma teve facilidade para acompanhar acompanhar as atividades da disciplina\\\\
Q03&A turma se empenhou em acompanhar as atividades da disciplina.\\\\
Q04&Você apresentou plano de ensino contendo: ementa, objetivos, metodologias de ensino, critérios de avaliação, conteúdos e bibliografia da disciplina.\\\\
Q05&Os conteúdos trabalhados na disciplina foram coerentes com os que foram apresentados no plano de ensino.\\\\
Q06&Você deu retorno sobre todas as avaliações realizadas: correções, discussão dos pontos principais, esclarecimentos sobre os erros cometidos.\\\\
Q07&Os recursos didáticos, audiovisuais, tecnológicos empregados pelo(a) docente contribuíram para que a aula fosse mais produtiva.\\\\
Q08&Você cumpriu a carga horária prevista no semestre para a disciplina.\\\\
Q09&O grau de dificuldade das avaliações foi compatível com as aulas ministradas.\\\\
Q10&A disciplina contribuiu na preparação para o exercício profissional.\\\\
Q11&A disciplina contribuiu para a aquisição de formação teórica/prática na área. \\\\
Q12&A disciplina contribuiu para a aquisição de cultura geral.\\\\
Q13&Você dispôs de horário de atendimento extra-classe.\\\\
Q14& As notas de todas as avaliações parciais foram divulgadas no SIGA até 3 (três) dias antes da data da avaliação subsequente, conforme determina o RAG. \\\\
Q15&Você manteve um comportamento respeitoso e cortês em relação aos discentes.
\end{tabularx}
}
\end{center}
\section{RESPOSTAS}
As questões podem ser respondidas com um número de 1 a 5 numa escala desde {\it Discordo Totalmente} (1) a {\it Concordo Totalmente} (5). Não sendo permitidas múltiplas respostas e sendo possível a alteração antes do envio do formulário. O valor 0 (zero) indica {\it Não se Aplica}.

Os nomes das disciplinas foram alterados para garantir a segurança da informações de modo que não seja possível a identificação das turmas, dos alunos  e dos professores que responderam a avaliação.

\begin{figure}[h]
\centering
\includegraphics[width=0.85\linewidth]{analise_geral_departamento_ANA_docentes.png}
\caption{\label{fig:analise_geral_departamento}            Panorama geral das respostas de todas as  disciplinas do departamenbto para as questões apresentadas.}
\end{figure}

\begin{figure}[h]
\centering
\includegraphics[width=0.999\linewidth]{matriz_corr__ANA_docentes.png}
\caption{\label{fig:corr_docentes}Correlação das respostas dos professores do departamento ANA. Entradas com em branco indicam que não havia informação suficiente para o cálculo das correlações.}
\end{figure}
\begin{figure}[h]
\centering
\includegraphics[width=0.485\linewidth]{analise_disciplina_departamento_ANA_5E77D919B7A28F0635631EAC6A175BCD_docentes.png}
\caption{\label{fig:analise_geral_departamento}                Distribuição das respostas para a disciplina 5E77D919B7A28F0635631EAC6A175BCD. O gráfico da avaliação dos alunos não foi mostrado  por um dos motivos:  a disciplina não foi avaliada pelos alunos ou somente um aluno realizou a avaliação. }
\end{figure}
\begin{figure}[h]
\centering
\includegraphics[width=0.485\linewidth]{analise_disciplina_departamento_ANA_9B90C2A83391770C3525AD0FC4DFF933_docentes.png}
\caption{\label{fig:analise_geral_departamento}                Distribuição das respostas para a disciplina 9B90C2A83391770C3525AD0FC4DFF933. O gráfico da avaliação dos alunos não foi mostrado  por um dos motivos:  a disciplina não foi avaliada pelos alunos ou somente um aluno realizou a avaliação. }
\end{figure}
\begin{figure}[h]
\centering
\includegraphics[width=0.485\linewidth]{analise_disciplina_departamento_ANA_CB2E8A0F0EBCF7CDCC5B46BEAE6587A0_docentes.png}
\caption{\label{fig:analise_geral_departamento}                Distribuição das respostas para a disciplina CB2E8A0F0EBCF7CDCC5B46BEAE6587A0. O gráfico da avaliação dos alunos não foi mostrado  por um dos motivos:  a disciplina não foi avaliada pelos alunos ou somente um aluno realizou a avaliação. }
\end{figure}
\begin{figure}[h]
\centering
\includegraphics[width=0.485\linewidth]{analise_disciplina_departamento_ANA_9E514DD262215D82AB957F80E4E9A67C_docentes.png}
\caption{\label{fig:analise_geral_departamento}                Distribuição das respostas para a disciplina 9E514DD262215D82AB957F80E4E9A67C. O gráfico da avaliação dos alunos não foi mostrado  por um dos motivos:  a disciplina não foi avaliada pelos alunos ou somente um aluno realizou a avaliação. }
\end{figure}
\begin{figure}[h]
\centering
\includegraphics[width=0.485\linewidth]{analise_disciplina_departamento_ANA_378F0222154EFC47E5EC36E80AF834A7_docentes.png}
\caption{\label{fig:analise_geral_departamento}                Distribuição das respostas para a disciplina 378F0222154EFC47E5EC36E80AF834A7. O gráfico da avaliação dos alunos não foi mostrado  por um dos motivos:  a disciplina não foi avaliada pelos alunos ou somente um aluno realizou a avaliação. }
\end{figure}
\begin{figure}[h]
\centering
\includegraphics[width=0.485\linewidth]{analise_disciplina_departamento_ANA_B4BD474015532B671FB188CB0702280A_docentes.png}
\caption{\label{fig:analise_geral_departamento}                Distribuição das respostas para a disciplina B4BD474015532B671FB188CB0702280A. O gráfico da avaliação dos alunos não foi mostrado  por um dos motivos:  a disciplina não foi avaliada pelos alunos ou somente um aluno realizou a avaliação. }
\end{figure}
\begin{figure}[h]
\centering
\includegraphics[width=0.485\linewidth]{analise_disciplina_departamento_ANA_D22984D3030A0ECA6CF5A358E7903F55_docentes.png}
\caption{\label{fig:analise_geral_departamento}                Distribuição das respostas para a disciplina D22984D3030A0ECA6CF5A358E7903F55. O gráfico da avaliação dos alunos não foi mostrado  por um dos motivos:  a disciplina não foi avaliada pelos alunos ou somente um aluno realizou a avaliação. }
\end{figure}
\begin{figure}[h]
\centering
\includegraphics[width=0.485\linewidth]{analise_disciplina_departamento_ANA_6B3AF075192A0BBA17D94F75D1F19EBF_docentes.png}
\caption{\label{fig:analise_geral_departamento}                Distribuição das respostas para a disciplina 6B3AF075192A0BBA17D94F75D1F19EBF. O gráfico da avaliação dos alunos não foi mostrado  por um dos motivos:  a disciplina não foi avaliada pelos alunos ou somente um aluno realizou a avaliação. }
\end{figure}
\begin{figure}[h]
\centering
\includegraphics[width=0.485\linewidth]{analise_disciplina_departamento_ANA_4B6CAC99858B22FA2F0721F636D80409_docentes.png}
\caption{\label{fig:analise_geral_departamento}                Distribuição das respostas para a disciplina 4B6CAC99858B22FA2F0721F636D80409. O gráfico da avaliação dos alunos não foi mostrado  por um dos motivos:  a disciplina não foi avaliada pelos alunos ou somente um aluno realizou a avaliação. }
\end{figure}
\begin{figure}[h]
\centering
\includegraphics[width=0.485\linewidth]{analise_disciplina_departamento_ANA_F0EB9C3AD7FFE1495BE75682AB17E293_docentes.png}
\caption{\label{fig:analise_geral_departamento}                Distribuição das respostas para a disciplina F0EB9C3AD7FFE1495BE75682AB17E293. O gráfico da avaliação dos alunos não foi mostrado  por um dos motivos:  a disciplina não foi avaliada pelos alunos ou somente um aluno realizou a avaliação. }
\end{figure}
\begin{figure}[h]
\centering
\includegraphics[width=0.485\linewidth]{analise_disciplina_departamento_ANA_A65CE96FABF2BA1DE436A2B2275F0AB7_docentes.png}
\caption{\label{fig:analise_geral_departamento}                Distribuição das respostas para a disciplina A65CE96FABF2BA1DE436A2B2275F0AB7. O gráfico da avaliação dos alunos não foi mostrado  por um dos motivos:  a disciplina não foi avaliada pelos alunos ou somente um aluno realizou a avaliação. }
\end{figure}

\end{document}
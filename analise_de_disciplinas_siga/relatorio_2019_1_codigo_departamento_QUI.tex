
\documentclass[a4paper,10pt]{article}
\usepackage{ucs}
\usepackage[utf8]{inputenc}
\usepackage[brazil]{babel}
\usepackage{fontenc}
\usepackage{graphicx,tabularx}
\usepackage[]{hyperref}
\sloppy
\date{Data de processamento: \today}\begin{document}
\author{Diretoria de Avaliação Institucional (DIAVI) \\ Universidade Federal de Juiz de Fora}

\title{RELATÓRIO DE RESULTADOS DA AVALIAÇÃO DE DEPARTAMENTO\\ Código do Departamento: QUI}
\maketitle
\section{INTRODUÇÃO}
Este relatório objetiva apresentar os resultados da avaliação de disciplinas do Departamento     de código QUI da Universidade Federal de Juiz de Fora, realizada pela     Diretoria de Avaliação Institucional e os encaminhamentos propostos a     partir destes resultados.

\begin{center}
\begin{tabularx}{\linewidth}{r|X}

Público-alvo:& Departamento  QUI\\

Período de coleta de dados:& 2019/1.\\

Forma de aplicação:& Online, por meio do SIGA.\\

Alunos   respondentes:& 9\\

Docentes respondentes:& 0\\

Disciplinas avaliadas pelos   Alunos:& 4\\

Disciplinas avaliadas pelos Docentes:& 0\\
\end{tabularx}
\end{center}

\section{MÉTODOS}
Este relatório se refere ao período 2019/1, com base em dados     coletados através da aplicação de instrumentos de avaliação via SIGA     implementados pela Diretoria de Avaliação Institucional (DIAVI) da UFJF, em atendimento     ao que estabelece a Lei Sinais e a Resolução Consu 13/2015 (UFJF),     com objetivo de contribuir para a avaliação própria do departamento QUI.    Foram aplicados um instrumento para discentes e outro para docentes, ambos contendo     23 questões versando sobre as disciplinas na modalidade presencial oferecidas pela UFJF no     referido período, visando, especificamente, coletar impressões sobre: atuação docente, atuação discente,     recursos empregados, qualidade da disciplina ministrada.     As respostas foram colhidas      com participação espontânea e garantia de    sigilo de participantes e avaliados.
\section{FORMULÁRIO}
As seguintes questões foram {\bf objeto de avaliação pelos discentes} através do SIGA.

\small{
\begin{center}
\begin{tabularx}{\linewidth}{l|X}
Q01&O plano de ensino apresentado pelo(a) docente contém: ementa, objetivos, metodologias de ensino, critérios de avaliação, cronograma, conteúdos e bibliografia da disciplina.\\\\
Q02&O(A) docente cumpriu o plano de ensino apresentado.\\\\
Q03&Os conteúdos foram ministrados de forma clara.\\\\
Q04&As atividades pedagógicas (aulas, atividades online e presenciais e etc) estimularam a sua participação.\\\\
Q05&Os recursos didáticos, audiovisuais e tecnológicos disponíveis contribuíram para o processo de ensino e aprendizagem.\\\\
Q06&Os critérios de avaliação foram compatíveis com os conteúdos ministrados.\\\\
Q07&O(A) docente deu retorno sobre todas as atividades realizadas.\\\\
Q08&A atuação e disponibilidade do(a) docente foram satisfatórias.\\\\
Q09&O(A) docente manteve um comportamento respeitoso e cortês em relação aos discentes.\\\\
Q10&A atuação e disponibilidade do(a) tutor(a) à distância foram satisfatórias.\\\\
Q11&O(A) tutor(a) à distância manteve um comportamento respeitoso e cortês em relação aos discentes.\\\\
Q12&As atividades propostas contribuíram para a aprendizagem dos conteúdos.\\\\
Q13&A disciplina contribuiu na preparação para o exercício profissional.\\\\
Q14&A disciplina contribuiu para a aquisição de cultura geral.\\\\
Q15&Sua motivação para estudar esta disciplina foi alta.\\\\
Q16&Você teve facilidade para acompanhar as atividades da disciplina.\\\\
Q17&Seu empenho durante a disciplina foi elevado.\\\\
Q18&A atuação do(a) tutor(a) presencial foi satisfatória.\\\\
Q19&A atuação do(a) assistente à docência foi satisfatória.\\\\
Q20&As estruturas física e tecnológica do polo foram compatíveis e atenderam ao desenvolvimento das atividades.\\\\
Q21&A atuação do(a) coordenador(a) de curso foi satisfatória.\\\\
Q22&O trabalho da secretaria do curso na UFJF foi satisfatório.\\\\
Q23&A atuação da UFJF em relação aos alunos e polo-UAB é satisfatória.
\end{tabularx}
\end{center}
}
As questões abaixo foram {\bf objeto de avaliação pelos docentes} através do SIGA.

\small{
\begin{center}
\begin{tabularx}{\linewidth}{c|X}
Q01&O plano de ensino apresentado pelo(a) docente contém: ementa, objetivos, metodologias de ensino, critérios de avaliação, cronograma, conteúdos e bibliografia da disciplina.\\\\
Q02&Você cumpriu o plano de ensino apresentado.\\\\
Q03&Os conteúdos foram ministrados de forma clara.\\\\
Q04&As atividades pedagógicas (aulas, atividades online e presenciais e etc) estimularam a participação discente.\\\\
Q05&Os recursos didáticos, audiovisuais e tecnológicos disponíveis contribuíram para o processo de ensino e aprendizagem.\\\\
Q06&Os critérios de avaliação foram compatíveis com os conteúdos ministrados.\\\\
Q07&Você deu retorno aos discentes sobre as atividades realizadas.\\\\
Q08&A sua atuação e disponibilidade foram satisfatórias.\\\\
Q09&Você manteve um comportamento respeitoso e cortês em relação aos discentes.\\\\
Q10&A atuação e disponibilidade do(a) tutor(a) à distância foram satisfatórias.\\\\
Q11&O(A) tutor(a) à distância manteve um comportamento respeitoso e cortês em relação aos discentes e ao docente..\\\\
Q12&As atividades propostas contribuíram para a aprendizagem dos conteúdos.\\\\
Q13&A disciplina contribuiu na preparação para o exercício profissional dos discentes\\\\
Q14&A disciplina contribuiu para a aquisição de cultura geral dos discentes.\\\\
Q15&A motivação da turma para estudar essa disciplina foi alta.\\\\
Q16&A turma teve facilidade para acompanhar acompanhar as atividades da disciplina\\\\
Q17&O empenho da turma durante a disciplina foi elevado.\\\\
Q18&A atuação do(a) tutor(a) presencial foi satisfatória.\\\\
Q19&A atuação do(a) assistente à docência foi satisfatória.\\\\
Q20&As estruturas física e tecnológica dos polos foram compatíveis e atenderam ao desenvolvimento das atividades.\\\\
Q21&A coordenação dos polos e as secretarias deram o apoio necessário para a realização das atividades.\\\\
Q22&A atuação do(a) coordenador(a) do curso foi satisfatória.\\\\
Q23&O trabalho da secretaria do curso na UFJF foi satisfatório.
\end{tabularx}
\end{center}
}
\section{RESPOSTAS}
As questões podem ser respondidas com um número de 1 a 5 numa escala que vai de {\it Discordo Totalmente} (1) a {\it Concordo Totalmente} (5). Não sendo permitidas múltiplas respostas e sendo possível a alteração antes do envio do formulário. O valor 0 (zero) indica {\it Não se Aplica}.

\subsection{Panorama de todas as disciplinas avaliadas}
\begin{figure}[h]
\centering
\includegraphics[width=0.85\linewidth]{analise_geral_departamento_QUI_ALUNO_TURMA.png}
\caption{\label{fig:analise_geral_departamento}            Panorama geral das respostas das disciplinas para as questões apresentadas.}
\end{figure}
\subsection{Distribuição das respostas para cada disciplina do departamento}
\begin{figure}[h]
\centering
\includegraphics[width=0.485\linewidth]{analise_disciplina_departamento_QUI_ALUNO_TURMA_0F5236DC8332C3DC40B52B2828AC19B8.png}
\caption{\label{fig:analise_geral_departamento}                Distribuição das respostas para a disciplina 0F5236DC8332C3DC40B52B2828AC19B8.}
\end{figure}
\begin{figure}[h]
\centering
\includegraphics[width=0.485\linewidth]{analise_disciplina_departamento_QUI_ALUNO_TURMA_2AB4F66B2B013BB5F52ECDCB4D705308.png}
\caption{\label{fig:analise_geral_departamento}                Distribuição das respostas para a disciplina 2AB4F66B2B013BB5F52ECDCB4D705308.}
\end{figure}
\begin{figure}[h]
\centering
\includegraphics[width=0.485\linewidth]{analise_disciplina_departamento_QUI_ALUNO_TURMA_90C5EB39DEC05FD27773545572112291.png}
\caption{\label{fig:analise_geral_departamento}                Distribuição das respostas para a disciplina 90C5EB39DEC05FD27773545572112291.}
\end{figure}
\begin{figure}[h]
\centering
\includegraphics[width=0.485\linewidth]{analise_disciplina_departamento_QUI_ALUNO_TURMA_9EEF660293506BEEC11B728729655CB9.png}
\caption{\label{fig:analise_geral_departamento}                Distribuição das respostas para a disciplina 9EEF660293506BEEC11B728729655CB9.}
\end{figure}

\end{document}
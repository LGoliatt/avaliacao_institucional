
\documentclass[a4paper,10pt]{article}
\usepackage{ucs}
\usepackage[utf8]{inputenc}
\usepackage[brazil]{babel}
\usepackage{a4wide}
\usepackage{fontenc}
\usepackage{times}
\usepackage{indentfirst}
\usepackage{graphicx,tabularx}
\usepackage[]{hyperref}
\sloppy
\date{Data de processamento: \today}\begin{document}
\author{Diretoria de Avaliação Institucional (DIAVI) \\ Universidade Federal de Juiz de Fora}

\title{RELATÓRIO DE RESULTADOS DA AVALIAÇÃO DE DEPARTAMENTO\\ Código do Departamento: ADMGV}
\maketitle
\section{INTRODUÇÃO}
Este relatório objetiva apresentar os resultados da avaliação de disciplinas do Departamento     de código ADMGV da Universidade Federal de Juiz de Fora, realizada pela     Diretoria de Avaliação Institucional e os encaminhamentos propostos a     partir destes resultados.

\begin{center}
\begin{tabularx}{\linewidth}{X|l}

Público-alvo:& Departamento  ADMGV\\

Período de coleta de dados:& 2019/3.\\

Forma de aplicação:& Online, por meio do SIGA.\\

Docentes respondentes:& 11\\

Número de disciplinas do departamento avaliadas pelos Docentes:& 21\\

Alunos   respondentes:& 107\\

Número de disciplinas do departamento  avaliadas pelos   Alunos:& 20\\
\end{tabularx}
\end{center}

Os resultados relativos as respostas dos discentes serão omitidos na ausência de alunos respondentes, ou na eventualidade de somente um aluno responder o questionário discente.
\section{MÉTODOS}
Este relatório se refere ao período 2019/3, com base em dados     coletados através da aplicação de instrumentos de avaliação via SIGA     implementados pela Diretoria de Avaliação Institucional (DIAVI) da UFJF, em atendimento     ao que estabelece a Lei Sinais e a Resolução Consu 13/2015 (UFJF),     com objetivo de contribuir para a avaliação própria do departamento ADMGV.    Foram aplicados um instrumento para discentes e outro para docentes, ambos contendo     15 questões versando sobre as disciplinas na modalidade presencial oferecidas pela UFJF no     referido período, visando, especificamente, coletar impressões sobre: atuação docente, atuação discente,     recursos empregados, qualidade da disciplina ministrada.     As respostas foram colhidas      com participação espontânea e garantia de    sigilo de participantes e avaliados.
\section{QUESTÕES APRESENTADAS NOS FORMULÁRIOS}
As seguintes questões foram {\bf objeto de avaliação pelos discentes} através do SIGA.

\begin{center}
\small{
\begin{tabularx}{\linewidth}{l|X}
Q01&Sua motivação para estudar esta disciplina foi alta.\\\\
Q02&Você teve facilidade para acompanhar as atividades da disciplina.\\\\
Q03&Seu empenho durante a disciplina foi elevado.\\\\
Q04&O plano de ensino apresentado pelo(a) docente contém: ementa, objetivos, metodologias de ensino, critérios de avaliação, conteúdos e bibliografia da disciplina.\\\\
Q05&Os conteúdos trabalhados na disciplina foram coerentes com os que foram apresentados no plano de ensino.\\\\
Q06&O(A) docente deu retorno sobre todas as avaliações realizadas: correções, discussão dos pontos principais, esclarecimentos sobre os erros cometidos.\\\\
Q07&Os recursos didáticos, audiovisuais, tecnológicos empregados pelo(a) docente contribuíram para que a aula fosse mais produtiva.\\\\
Q08&O(A) docente cumpriu a carga horária prevista no semestre para a disciplina.\\\\
Q09&O grau de dificuldade das avaliações foi compatível com as aulas ministradas.\\\\
Q10&A disciplina contribuiu na preparação para o exercício profissional.\\\\
Q11&A disciplina contribuiu para a aquisição de formação teórica/prática na área. \\\\
Q12&A disciplina contribuiu para a aquisição de cultura geral.\\\\
Q13&O(A) docente dispôs de horários de atendimento extra-classe.\\\\
Q14& As notas de todas as avaliações parciais foram divulgadas no SIGA até 3 (três) dias antes da data da avaliação subsequente, conforme determina o RAG. \\\\
Q15&O(A) docente manteve um comportamento respeitoso e cortês em relação aos discentes.
\end{tabularx}
}
\end{center}
As questões abaixo foram {\bf objeto de avaliação pelos docentes} através do SIGA.

\begin{center}
\small{
\begin{tabularx}{\linewidth}{c|X}
Q01&A turma se mostrou motivada para estudar a disciplina.\\\\
Q02&A turma teve facilidade para acompanhar acompanhar as atividades da disciplina\\\\
Q03&A turma se empenhou em acompanhar as atividades da disciplina.\\\\
Q04&Você apresentou plano de ensino contendo: ementa, objetivos, metodologias de ensino, critérios de avaliação, conteúdos e bibliografia da disciplina.\\\\
Q05&Os conteúdos trabalhados na disciplina foram coerentes com os que foram apresentados no plano de ensino.\\\\
Q06&Você deu retorno sobre todas as avaliações realizadas: correções, discussão dos pontos principais, esclarecimentos sobre os erros cometidos.\\\\
Q07&Os recursos didáticos, audiovisuais, tecnológicos empregados pelo(a) docente contribuíram para que a aula fosse mais produtiva.\\\\
Q08&Você cumpriu a carga horária prevista no semestre para a disciplina.\\\\
Q09&O grau de dificuldade das avaliações foi compatível com as aulas ministradas.\\\\
Q10&A disciplina contribuiu na preparação para o exercício profissional.\\\\
Q11&A disciplina contribuiu para a aquisição de formação teórica/prática na área. \\\\
Q12&A disciplina contribuiu para a aquisição de cultura geral.\\\\
Q13&Você dispôs de horário de atendimento extra-classe.\\\\
Q14& As notas de todas as avaliações parciais foram divulgadas no SIGA até 3 (três) dias antes da data da avaliação subsequente, conforme determina o RAG. \\\\
Q15&Você manteve um comportamento respeitoso e cortês em relação aos discentes.
\end{tabularx}
}
\end{center}
\section{RESPOSTAS}
As questões podem ser respondidas com um número de 1 a 5 numa escala desde {\it Discordo Totalmente} (1) a {\it Concordo Totalmente} (5). Não sendo permitidas múltiplas respostas e sendo possível a alteração antes do envio do formulário. O valor 0 (zero) indica {\it Não se Aplica}.

Os nomes das disciplinas foram alterados para garantir a segurança da informações de modo que não seja possível a identificação das turmas, dos alunos  e dos professores que responderam a avaliação.

\begin{figure}[h]
\centering
\includegraphics[width=0.85\linewidth]{analise_geral_departamento_ADMGV_docentes.png}
\includegraphics[width=0.85\linewidth]{analise_geral_departamento_ADMGV_alunos.png}
\caption{\label{fig:analise_geral_departamento}            Panorama geral das respostas de todas as  disciplinas do departamenbto para as questões apresentadas.}
\end{figure}

\begin{figure}[h]
\centering
\includegraphics[width=0.999\linewidth]{matriz_corr__ADMGV_docentes.png}
\caption{\label{fig:corr_docentes}Correlação das respostas dos professores do departamento ADMGV. Entradas com em branco indicam que não havia informação suficiente para o cálculo das correlações.}
\end{figure}

\begin{figure}[h]
\centering
\includegraphics[width=0.999\linewidth]{matriz_corr__ADMGV_alunos.png}
\caption{\label{fig:corr_alunos}Correlação das respostas dos alunos para o departamento ADMGV. Entradas com em branco indicam que não havia informação suficiente para o cálculo das correlações.}
\end{figure}
\begin{figure}[h]
\centering
\includegraphics[width=0.485\linewidth]{analise_disciplina_departamento_ADMGV_1C4C1CFE41D0DAF97B0FC3A16DF64847_docentes.png}
\includegraphics[width=0.485\linewidth]{analise_disciplina_departamento_ADMGV_1C4C1CFE41D0DAF97B0FC3A16DF64847_alunos.png}
\caption{\label{fig:analise_geral_departamento}                Distribuição das respostas para a disciplina 1C4C1CFE41D0DAF97B0FC3A16DF64847. }
\end{figure}
\begin{figure}[h]
\centering
\includegraphics[width=0.485\linewidth]{analise_disciplina_departamento_ADMGV_006793483C8A4C82F6F56964B1243CE0_docentes.png}
\includegraphics[width=0.485\linewidth]{analise_disciplina_departamento_ADMGV_006793483C8A4C82F6F56964B1243CE0_alunos.png}
\caption{\label{fig:analise_geral_departamento}                Distribuição das respostas para a disciplina 006793483C8A4C82F6F56964B1243CE0. }
\end{figure}
\begin{figure}[h]
\centering
\includegraphics[width=0.485\linewidth]{analise_disciplina_departamento_ADMGV_4B0AE0104DAE0C45C964E6BE10C98912_docentes.png}
\includegraphics[width=0.485\linewidth]{analise_disciplina_departamento_ADMGV_4B0AE0104DAE0C45C964E6BE10C98912_alunos.png}
\caption{\label{fig:analise_geral_departamento}                Distribuição das respostas para a disciplina 4B0AE0104DAE0C45C964E6BE10C98912. }
\end{figure}
\begin{figure}[h]
\centering
\includegraphics[width=0.485\linewidth]{analise_disciplina_departamento_ADMGV_7BEB5D45399FE1B01E55B01B08BC4695_docentes.png}
\includegraphics[width=0.485\linewidth]{analise_disciplina_departamento_ADMGV_7BEB5D45399FE1B01E55B01B08BC4695_alunos.png}
\caption{\label{fig:analise_geral_departamento}                Distribuição das respostas para a disciplina 7BEB5D45399FE1B01E55B01B08BC4695. }
\end{figure}
\begin{figure}[h]
\centering
\includegraphics[width=0.485\linewidth]{analise_disciplina_departamento_ADMGV_3F1269DD539D7EE2D7BAB461EE387EF0_docentes.png}
\includegraphics[width=0.485\linewidth]{analise_disciplina_departamento_ADMGV_3F1269DD539D7EE2D7BAB461EE387EF0_alunos.png}
\caption{\label{fig:analise_geral_departamento}                Distribuição das respostas para a disciplina 3F1269DD539D7EE2D7BAB461EE387EF0. }
\end{figure}
\begin{figure}[h]
\centering
\includegraphics[width=0.485\linewidth]{analise_disciplina_departamento_ADMGV_BB232703A608E8D12B7E6629A83A0DB6_docentes.png}
\includegraphics[width=0.485\linewidth]{analise_disciplina_departamento_ADMGV_BB232703A608E8D12B7E6629A83A0DB6_alunos.png}
\caption{\label{fig:analise_geral_departamento}                Distribuição das respostas para a disciplina BB232703A608E8D12B7E6629A83A0DB6. }
\end{figure}
\begin{figure}[h]
\centering
\includegraphics[width=0.485\linewidth]{analise_disciplina_departamento_ADMGV_A1F1D7BDE047163027A8C047133AF7DB_docentes.png}
\includegraphics[width=0.485\linewidth]{analise_disciplina_departamento_ADMGV_A1F1D7BDE047163027A8C047133AF7DB_alunos.png}
\caption{\label{fig:analise_geral_departamento}                Distribuição das respostas para a disciplina A1F1D7BDE047163027A8C047133AF7DB. }
\end{figure}
\begin{figure}[h]
\centering
\includegraphics[width=0.485\linewidth]{analise_disciplina_departamento_ADMGV_85F89F8D7638EA5D7E8B547DA7B6348F_docentes.png}
\includegraphics[width=0.485\linewidth]{analise_disciplina_departamento_ADMGV_85F89F8D7638EA5D7E8B547DA7B6348F_alunos.png}
\caption{\label{fig:analise_geral_departamento}                Distribuição das respostas para a disciplina 85F89F8D7638EA5D7E8B547DA7B6348F. }
\end{figure}
\begin{figure}[h]
\centering
\includegraphics[width=0.485\linewidth]{analise_disciplina_departamento_ADMGV_D560D0660D60E6A7AA7E0069D9361332_docentes.png}
\includegraphics[width=0.485\linewidth]{analise_disciplina_departamento_ADMGV_D560D0660D60E6A7AA7E0069D9361332_alunos.png}
\caption{\label{fig:analise_geral_departamento}                Distribuição das respostas para a disciplina D560D0660D60E6A7AA7E0069D9361332. }
\end{figure}
\begin{figure}[h]
\centering
\includegraphics[width=0.485\linewidth]{analise_disciplina_departamento_ADMGV_32D4D31CADBE51DF0E275741854181D4_docentes.png}
\includegraphics[width=0.485\linewidth]{analise_disciplina_departamento_ADMGV_32D4D31CADBE51DF0E275741854181D4_alunos.png}
\caption{\label{fig:analise_geral_departamento}                Distribuição das respostas para a disciplina 32D4D31CADBE51DF0E275741854181D4. }
\end{figure}
\begin{figure}[h]
\centering
\includegraphics[width=0.485\linewidth]{analise_disciplina_departamento_ADMGV_58173670E23308C8DED561E4257CF1B7_docentes.png}
\includegraphics[width=0.485\linewidth]{analise_disciplina_departamento_ADMGV_58173670E23308C8DED561E4257CF1B7_alunos.png}
\caption{\label{fig:analise_geral_departamento}                Distribuição das respostas para a disciplina 58173670E23308C8DED561E4257CF1B7. }
\end{figure}
\begin{figure}[h]
\centering
\includegraphics[width=0.485\linewidth]{analise_disciplina_departamento_ADMGV_10A5E2BCAF1D5CC7F4CC659C68906DB5_docentes.png}
\includegraphics[width=0.485\linewidth]{analise_disciplina_departamento_ADMGV_10A5E2BCAF1D5CC7F4CC659C68906DB5_alunos.png}
\caption{\label{fig:analise_geral_departamento}                Distribuição das respostas para a disciplina 10A5E2BCAF1D5CC7F4CC659C68906DB5. }
\end{figure}
\begin{figure}[h]
\centering
\includegraphics[width=0.485\linewidth]{analise_disciplina_departamento_ADMGV_3E2AB6086DC76CAE08B9702D61B3ECB7_docentes.png}
\includegraphics[width=0.485\linewidth]{analise_disciplina_departamento_ADMGV_3E2AB6086DC76CAE08B9702D61B3ECB7_alunos.png}
\caption{\label{fig:analise_geral_departamento}                Distribuição das respostas para a disciplina 3E2AB6086DC76CAE08B9702D61B3ECB7. }
\end{figure}
\begin{figure}[h]
\centering
\includegraphics[width=0.485\linewidth]{analise_disciplina_departamento_ADMGV_1CE2D34203769A3F376434C50943DFFD_docentes.png}
\includegraphics[width=0.485\linewidth]{analise_disciplina_departamento_ADMGV_1CE2D34203769A3F376434C50943DFFD_alunos.png}
\caption{\label{fig:analise_geral_departamento}                Distribuição das respostas para a disciplina 1CE2D34203769A3F376434C50943DFFD. }
\end{figure}
\begin{figure}[h]
\centering
\includegraphics[width=0.485\linewidth]{analise_disciplina_departamento_ADMGV_D337A4F73125FC027D86FAC19B7582EC_docentes.png}
\includegraphics[width=0.485\linewidth]{analise_disciplina_departamento_ADMGV_D337A4F73125FC027D86FAC19B7582EC_alunos.png}
\caption{\label{fig:analise_geral_departamento}                Distribuição das respostas para a disciplina D337A4F73125FC027D86FAC19B7582EC. }
\end{figure}
\begin{figure}[h]
\centering
\includegraphics[width=0.485\linewidth]{analise_disciplina_departamento_ADMGV_2F824088AACEF61D3BA771DEB0604231_docentes.png}
\includegraphics[width=0.485\linewidth]{analise_disciplina_departamento_ADMGV_2F824088AACEF61D3BA771DEB0604231_alunos.png}
\caption{\label{fig:analise_geral_departamento}                Distribuição das respostas para a disciplina 2F824088AACEF61D3BA771DEB0604231. }
\end{figure}
\begin{figure}[h]
\centering
\includegraphics[width=0.485\linewidth]{analise_disciplina_departamento_ADMGV_0859E89580699E1730C1B8614ADB82B6_docentes.png}
\includegraphics[width=0.485\linewidth]{analise_disciplina_departamento_ADMGV_0859E89580699E1730C1B8614ADB82B6_alunos.png}
\caption{\label{fig:analise_geral_departamento}                Distribuição das respostas para a disciplina 0859E89580699E1730C1B8614ADB82B6. }
\end{figure}
\begin{figure}[h]
\centering
\includegraphics[width=0.485\linewidth]{analise_disciplina_departamento_ADMGV_24BE8ABAB04D4C7D45F2C3F841A5F623_docentes.png}
\includegraphics[width=0.485\linewidth]{analise_disciplina_departamento_ADMGV_24BE8ABAB04D4C7D45F2C3F841A5F623_alunos.png}
\caption{\label{fig:analise_geral_departamento}                Distribuição das respostas para a disciplina 24BE8ABAB04D4C7D45F2C3F841A5F623. }
\end{figure}
\begin{figure}[h]
\centering
\includegraphics[width=0.485\linewidth]{analise_disciplina_departamento_ADMGV_0083EDB2CEABD886DB53B3D019510B34_docentes.png}
\includegraphics[width=0.485\linewidth]{analise_disciplina_departamento_ADMGV_0083EDB2CEABD886DB53B3D019510B34_alunos.png}
\caption{\label{fig:analise_geral_departamento}                Distribuição das respostas para a disciplina 0083EDB2CEABD886DB53B3D019510B34. }
\end{figure}
\begin{figure}[h]
\centering
\includegraphics[width=0.485\linewidth]{analise_disciplina_departamento_ADMGV_40E7875422E8CB68FD2EFB3FD75954E6_docentes.png}
\caption{\label{fig:analise_geral_departamento}                Distribuição das respostas para a disciplina 40E7875422E8CB68FD2EFB3FD75954E6. O gráfico da avaliação dos alunos não foi mostrado  por um dos motivos:  a disciplina não foi avaliada pelos alunos ou somente um aluno realizou a avaliação. }
\end{figure}
\begin{figure}[h]
\centering
\includegraphics[width=0.485\linewidth]{analise_disciplina_departamento_ADMGV_5B13F373CD86BD562FEE363ED4A90AA0_docentes.png}
\includegraphics[width=0.485\linewidth]{analise_disciplina_departamento_ADMGV_5B13F373CD86BD562FEE363ED4A90AA0_alunos.png}
\caption{\label{fig:analise_geral_departamento}                Distribuição das respostas para a disciplina 5B13F373CD86BD562FEE363ED4A90AA0. }
\end{figure}

\end{document}